\documentclass[a4paper,11pt]{article}
\usepackage{fontspec}
\usepackage{xltxtra}
\usepackage{xgreek}
\usepackage{xunicode}
\usepackage{amsmath}
\usepackage{listings}
\usepackage{xcolor}
\usepackage{graphicx}
%\lstset{
%    language=Python,
%    basicstyle=\ttfamily\small,
%    keywordstyle=\color{blue},
%    stringstyle=\color{red},
%    commentstyle=\color{gray},
%    frame=single,
%    numbers=left,
%    numberstyle=\tiny,
%    stepnumber=1,
%    numbersep=5pt
%}
\title{Triadic-Optimization}
\author{Ioannis Anagnostakis \\
rizitis@gmail.com}

\date{Jun 5, 2024} 
\setmainfont{DejaVu Sans}

\begin{document}
\maketitle
\section{Introduction (Εισαγωγή)}
We execute the \texttt{createnums.py} script, which generates a file named \texttt{numbers.txt} in the current directory. This file contains one billion lines, each containing a number in the range $[1,1,000,000,000]$.

Εκτελούμε το script \texttt{createnums.py}, το οποίο δημιουργεί ένα αρχείο \texttt{numbers.txt} στον τρέχοντα κατάλογο. Αυτό το αρχείο περιέχει ένα δισεκατομμύριο γραμμές, κάθε μία με έναν αριθμό από το $[1,1,000,000,000]$.

\section{Processing Steps (Βήματα Επεξεργασίας)}
Next, we run either \texttt{teliko\_nums.py} or \texttt{teliko\_nums2.py}. These scripts perform the following operations:

Στη συνέχεια, εκτελούμε είτε το \texttt{teliko\_nums.py} είτε το \texttt{teliko\_nums2.py}. Αυτά τα scripts εκτελούν τις ακόλουθες διαδικασίες:

\begin{enumerate}
    \item Randomly shuffle the lines in \texttt{numbers.txt}. (Ανακατεύουν τυχαία τις γραμμές στο \texttt{numbers.txt}).
    \item Count the total number of lines ($10^9$). (Μετρούν το συνολικό αριθμό γραμμών $10^9$).
    \item Randomly divide the lines into three segments. (Διαχωρίζουν τυχαία τις γραμμές σε τρία τμήματα).
    \item Compute the mean for each segment: (Υπολογίζουν τον μέσο όρο για κάθε τμήμα:)
    \begin{equation}
        \mu_i = \frac{\sum_{j=1}^{N_i} x_j}{N_i},
    \end{equation}
    where $N_i$ is the number of lines in segment $i$ and $x_j$ is the number in line $j$. (όπου $N_i$ είναι ο αριθμός γραμμών στο τμήμα $i$ και $x_j$ είναι ο αριθμός στη γραμμή $j$).
\end{enumerate}

\section{Results (Αποτελέσματα)}
The resulting means exhibit small variations:

Οι προκύπτοντες μέσοι όροι εμφανίζουν μικρές αποκλίσεις:

\begin{verbatim}
python3 teliko_nums.py
The mean of the first segment is: 499985344.58548886
The mean of the second segment is: 499997023.58038485
The mean of the third segment is: 500017633.33407336
\end{verbatim}

or (ή)

\begin{verbatim}
python3 teliko_nums2.py
The mean of the first segment is: 499981295.966643
The mean of the second segment is: 500010723.68831867
The mean of the third segment is: 500007981.8450144
\end{verbatim}

\section{Conclusion (Συμπέρασμα)}
At such large scales, randomness ensures the accuracy of the mean. This is significant because:

Σε τέτοιες μεγάλες κλίμακες, η τυχαιότητα εξασφαλίζει την ακρίβεια του μέσου όρου. Αυτό είναι σημαντικό επειδή:

\begin{itemize}
    \item A correct mean indicates a representative sample from the dataset. (Ένας σωστός μέσος όρος δείχνει ένα αντιπροσωπευτικό δείγμα από το dataset).
    \item The dataset, originally one billion numbers, could instead be a collection of files (e.g., text files, images, etc.). (Το dataset, αρχικά ένα δισεκατομμύριο αριθμοί, θα μπορούσε να είναι μια συλλογή αρχείων όπως κείμενα, εικόνες κ.λπ.).
    \item Using this method, we can train a model with only $\frac{1}{3}$ of the dataset and allocate less than $10\%$ for testing and validation. (Με αυτή τη μέθοδο, μπορούμε να εκπαιδεύσουμε ένα μοντέλο με μόνο το $\frac{1}{3}$ του dataset και να διαθέσουμε λιγότερο από $10\%$ για δοκιμή και επικύρωση).
\end{itemize}

The methodology for selecting the $10\%$ subset will be discussed separately. (Η μεθοδολογία για την επιλογή του $10\%$ συνόλου θα συζητηθεί ξεχωριστά.)
{}
\section{Summary - Final Stage (Σύνοψη - Τελικό Στάδιο)}
We execute the \texttt{createnums.py} script, which generates a file named \texttt{numbers.txt} in the current directory. This file contains one billion lines, each containing a number in the range $[1,1,000,000,000]$.

Εκτελούμε το script \texttt{createnums.py}, το οποίο δημιουργεί ένα αρχείο \texttt{numbers.txt} στον τρέχοντα κατάλογο. Αυτό το αρχείο περιέχει ένα δισεκατομμύριο γραμμές, κάθε μία με έναν αριθμό από το $[1,1,000,000,000]$.

\section{Processing Steps (Βήματα Επεξεργασίας)}
Next, we run either \texttt{teliko\_nums.py} or \texttt{teliko\_nums2.py}. These scripts perform the following operations:

Στη συνέχεια, εκτελούμε είτε το \texttt{teliko\_nums.py} είτε το \texttt{teliko\_nums2.py}. Αυτά τα scripts εκτελούν τις ακόλουθες διαδικασίες:

\begin{enumerate}
    \item Randomly shuffle the lines in \texttt{numbers.txt}. (Ανακατεύουν τυχαία τις γραμμές στο \texttt{numbers.txt}).
    \item Count the total number of lines ($10^9$). (Μετρούν το συνολικό αριθμό γραμμών $10^9$).
    \item Randomly divide the lines into three segments. (Διαχωρίζουν τυχαία τις γραμμές σε τρία τμήματα).
    \item Compute the mean for each segment: (Υπολογίζουν τον μέσο όρο για κάθε τμήμα:)
    \begin{equation}
        \mu_i = \frac{\sum_{j=1}^{N_i} x_j}{N_i},
    \end{equation}
    where $N_i$ is the number of lines in segment $i$ and $x_j$ is the number in line $j$. (όπου $N_i$ είναι ο αριθμός γραμμών στο τμήμα $i$ και $x_j$ είναι ο αριθμός στη γραμμή $j$).
\end{enumerate}
{}
\section{Dataset Processing (Επεξεργασία Συνόλου Δεδομένων)}
Let's move on to practice now. (Καλή η θεωρεία, ας περάσουμε στην πράξη τώρα.)

Assume that the \texttt{all\_data} folder represents the dataset with all the files we want to use to train the model. (Ας υποθέσουμε ότι ο φάκελος \texttt{all\_data} είναι το dataset με όλα τα αρχεία που θέλουμε να εκπαιδεύσουμε το μοντέλο.)

Running \texttt{step-4teliko\_arheia.py} will perform the following: (Εκτελώντας το \texttt{step-4teliko\_arheia.py} θα γίνουν τα παρακάτω:)

\begin{enumerate}
    \item Read the names of the files from the \texttt{all\_data} folder. (Ανάγνωση των ονομάτων των αρχείων από τον φάκελο \texttt{all\_data})
    \item Randomly shuffle the files. (Τυχαία ανάκατεμα των αρχείων)
    \item Calculate the total number of files. (Υπολογισμός του συνολικού αριθμού αρχείων)
    \item Calculate the size of each segment (33\%). (Υπολογισμός του μεγέθους κάθε τμήματος (33\%))
    \item Create folders for the 3 segments. (Δημιουργία φακέλων για τα 3 τμήματα)
    \item Separate the files and move them to the respective folders. (Διαχωρισμός των αρχείων και μεταφορά στους αντίστοιχους φακέλους.)
\end{enumerate}

Next, we run \texttt{step-5create\_test.py}, where the following actions take place: (Στη συνέχεια τρέχουμε το \texttt{step-5create\_test.py}, όπου εκεί θα γίνουν τα παρακάτω:)

\begin{enumerate}
    \item Create the folders \texttt{merged}, \texttt{test\_data}, and \texttt{val\_data}. (Θα δημιουργηθούν οι φάκελοι \texttt{merged}, \texttt{test\_data} και \texttt{val\_data})
    \item Merge the \texttt{data1} and \texttt{data3} folders into one folder named \texttt{merged}. (Οι φάκελοι \texttt{data1} και \texttt{data3} θα συγχωνευτούν σε ένα φάκελο \texttt{merged}.)
    \item Count the files in the \texttt{merged} folder and randomly select one fifth for testing and another one fifth for validation. (Μετράμε τα αρχεία στον φάκελο \texttt{merged} και τυχαία διαλέγουμε το 1/5 για τεστ και άλλο 1/5 για επικύρωση.)
\end{enumerate}

Now, we can train our model and evaluate if the \textbf{Triadic Optimization} approach is effective! (Αυτό ήταν! Μπορούμε τώρα να δοκιμάσουμε να εκπαιδεύσουμε το μοντέλο μας και να αξιολογήσουμε αν η \textbf{Triadic Optimization} αξίζει τον κόπο!)
\end{document}